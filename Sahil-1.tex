\documentclass{article}
% Comment the following line to NOT allow the usage of umlauts
\usepackage[utf8]{inputenc}
% Uncomment the following line to allow the usage of graphics (.png, .jpg)
%\usepackage{graphicx}

% Start the document
\begin{document}

% Create a new 1st level heading
\section{WHAT IS DISRUPTIVE INNOVATION}

Since Clayton Christensen's landmark book "The Innovator's Dilemma" introduced the notion of disruptive innovation, it has gotten a lot of attention. Its meaning, however, is frequently misinterpreted or misapplied. Christensen examined the notion in the pages of the Harvard Business Review, more than a decade after the book's release. Disruptive innovation, according to him and his co-authors, begins with items that establish low-end or new-market footholds before aggressively moving upwards and eliminating established competitors.


To put it another way, disruption occurs when a corporation recognises and meets an unmet (and sometimes latent) demand. Take, for example, the vehicle. Cars were not a disruptive breakthrough in the late 1800s. They were high-end transportation goods that cost a lot of money. The majority of people continued to travel by horse-drawn carriages, which were more economical. With the introduction of the Ford Motor Company's "Model T" in 1908, the industry was really disrupted.


This lower-cost vehicle revolutionised the transportation sector by making autos an affordable alternative to horse-drawn carriages, resulting in the creation of a whole new class of customers. Netflix is a modern example of disruptive innovation. Netflix didn't appear to challenge Blockbuster, the market leader in movie rentals, at first, because it catered to a limited set of consumers with its online interface and mail-order delivery. Netflix, on the other hand, was able to totally rethink its economic model because to the arrival of streaming technologies. Netflix was able to immediately dethrone Blockbuster due to its ability to pivot quickly, as well as the convenience, ease-of-use, and cheap price point of their service.


Despite the fact that disruption has been present for a long time, it is now moving at a faster speed. Every two weeks, a publicly traded firm is replaced by a newcomer.

\section{WHAT LEADS TO DISRUPTIVE INNOVATION}

The disruption of industries and marketplaces has been aided by technological advancements and information digitalization. It would be a mistake, though, to blame disruption only on technology. Disruptive innovation is fueled by a number of additional reasons, regardless of industry. One or more of the following dynamics are common in industries ripe for disruption.

\subsection{COMPLEX EXPERIENCES}

Few sectors are as complicated as healthcare. "Healthcare is a complex adaptive system, meaning that the system's performance and behaviour change over time and cannot be completely understood by simply knowing about the individual components," says Jeffrey Braithwaite, professor of health systems research and president of the International Society for Quality in Health Care.


No other industry or sector compares in terms of scope and diversity—complex funding structures, various moving elements, complex customers with different demands, and a plethora of alternatives and treatments for every given person's needs." Simple healthcare requirements may often lead to a tangle of appointments, practitioners, treatments, and locations.


When this happens, customers may feel that dealing with the healthcare system isn't worth their time or money, perhaps putting their health at risk. In this context, Minute Clinic was established to give easy, high-quality access to regular healthcare. Consumers can walk in without an appointment and receive care from a nurse practitioner or physician's assistant at these clinics, which are located inside retail chains like CVS and Target.


Sports physicals, flu vaccines, and the detection and treatment of common ailments like bronchitis and strep throat are all available. This type of care has been well-received.
Since the first retail clinic opened more than 20 years ago, the number of such clinics in the United States has grown to about 3,000. While Minute Clinics are unlikely to replace primary care professionals, the ease of receiving a flu vaccination while shopping might be the start of a disruptive innovation that leads to more alternative care locations.

\subsection{CONSUMER CONFUSION}

The insurance industry has a lengthy history that can be traced all the way back to ancient Babylon and is documented in the Code of Hammurabi. Despite the societal advantages of risk management, participants and regulators have frequently sought for more openness in product definition, pricing, and claims processing.


Insurance products are typically seen as complicated and difficult to understand, with contracts full of legalese and jargon. The insurance sector appears to be more focused on the business side of things than on client demands at times.


Lemonade is a firm with a mission to make insurance more appealing. It is designed to be quick and easy to use––90 seconds to be covered, three minutes to get paid––making the entire customer lifecycle experience nearly frictionless for customers. Furthermore, by giving any remaining premiums to a charity of the customer's choosing, the organisation has transformed insurance from a required risk management tool to a societal benefit.

\subsection{REDUNDANT INTERMEDIARIES}

Most retail travel companies have succumbed to disruptive innovation, which was formerly a mainstay of strip mall stores. The travel agent used to serve as a middleman, scouting costs and planning itineraries for travellers. Travel agents have become obsolete as a result of extensive internet access and sites like Expedia, Travelocity, and Orbitz.


Consumers can compare airline fares and hotel accommodations with just a few clicks, allowing them to book with simplicity. Travel agents are no longer the sole place to learn about various places and activities. Agents' competence has been eroded by the growth of review sites and suggestions from other tourists on social media. The travel business will continue to change as "anywhere, anytime" technology, as well as AI virtual assistants, grow.


With improved convenience and decreased travel costs, this disruption will almost probably continue to favour the ordinary customer.

\subsection{LIMITED ACCESS}

There are many of financial management organisations to choose from throughout the world. Many financial advisers may choose to deal with high-net-worth customers who want a variety of services and are ready to pay the costs associated with high-touch service. As a result of this strategy, a substantial number of younger and/or less wealthy potential investors are asking for assistance in managing their money in a simple, easy, and low-cost manner. Robo-advisors are stepping in to fill the void.


Financial management services are provided by robo-advisors, which are algorithm-based automated systems. Although using technology to make investing easier isn't new, making it available to consumers is. Robo-advisor platforms like E Trade, TD Ameritrade, and Robinhood, like previous disruptive developments, provide services for a fraction of the cost of conventional human advisers while also increasing convenience.


The disruption of wealth management may be here to stay as market demographics evolve and younger, technologically savvy employees build fortunes.

\section{WHO BENEFITS FROM DISRUPTIVE MODELS}

Disruptive innovation benefits consumers without a doubt. Disruptive products address their demands in a market that may have previously been unable to serve them due to their ease of access, convenience, and cheaper costs.


An argument may be made, somewhat counterintuitively, that disruption benefits all industry players. While established organisations may face some disruption, turbulence drives them to reconsider their business models and refocus on their core competencies.


They now have new chances to improve existing goods, invest in their employees, and guarantee that they are meeting the needs of their customers.

\section{HOW WILL HEALTHCARE GET DISRUPTED}

A "wicked problem" is a challenge that is difficult or impossible to address in planning and policy. Incomplete knowledge, the amount of parties and/or viewpoints involved, and how the problem overlaps with other problems are all factors that go into a wicked problem.


The healthcare business is going to be disrupted from an economic aspect.
Over the previous decade, the cost of healthcare per person has increased by about 4 percent, surpassing GDP growth of two to three percent. This is an extremely unsustainable situation. The great financial success of payers, providers, and pharmaceutical companies, on the other hand, gives little motivation for reform.

\section{THE RESULT OF HEALTHCARE DISRUPTION}

While previous success is no guarantee of future results, it's a near-certainty that the disruptive innovation process will enhance healthcare, just as it has in so many other industries. Health and economic benefits will extend to stakeholders and beyond as innovative products fulfil customer demand.


This isn't to say that present industry participants aren't important. Hospitals, insurers, providers, and others in the healthcare ecosystem can benefit from disruptive innovation, even if they did not initiate it. Established firms are starting to take ownership of innovation, creating leadership roles in some cases to assist improve their product offerings.


Thankfully, established digital health providers are already discovering methods to provide simple, cost-effective treatment on demand in the face of upheaval. They will not, however, be able to overhaul the sector on their own. In order for society to realise the benefits of consumer-focused healthcare in the years to come, smart policy, private entrepreneurship, and ongoing backing of entrepreneurs will be required.

% Uncomment the following two lines if you want to have a bibliography
%\bibliographystyle{alpha}
%\bibliography{document}

\end{document}